\documentclass{tufte-handout}
\usepackage[utf8]{inputenc}
\usepackage{tikz}
\usepackage{amsmath}

\usepackage{color}
\newcommand{\red}[1]{{\color{red} #1}}
\usepackage{booktabs}
\begin{document}
\section{Red Scare! Report}

by Group C.

\subsection{Results}

The following table gives my results for all graphs of at least 500 vertices.

\medskip
\begin{tabular}{lrrrrrr}
  \toprule
  Instance name & $n$ & A & F & M & N & S \\
  \midrule
  rusty-5762 & 5,762 & true & 16 & -- & ? & 5 \\
  wall-p-10000 & 10,000 &\\	
  $\vdots$\\
  \bottomrule
\end{tabular}
\medskip

The columns are for the problems Alternate, Few, Many, None, and Some.
The table entries either give the answer, or contain `?' for those cases where I was unable to find a solution within reasonable time.
For those questions where there is a reason for my inability to find a good algorithm (because the problem is hard), I wrote `?!'.

For the complete table of all results, see the tab-separted text file {\tt results.txt}.

\subsection{Methods}

\subsection{None}

For problem N, we solved each instance $G$ by removing all red vertices from $G$ and computing shortest path from $start$ to $finish$  by using Dijkstra's algorithm. 
The running time of this algorithm is $O(N + N^2)$ where $N$ is a number of vertices. Our implementation spends $0.0258$ seconds on the instance common-1-5000.txt with $n=5000$.

\subsection{Some}


\subsection{Many}
Why cannot solve all graphs -> longest path problem - can solve only directed acyclic graphs

\subsection{Few}

For problem F, we solved each instance $G$ by putting 

\subsection{Alternate}

For  problem A, we solved each instance $G$ by removing all edges that are connected with vertices that are both red or black and and computing shortest path from $start$ to $finish$ by using Dijkstra's algorithm.
The running time of this algorithm is $O(E + V^2)$ where  $E$ is number of edges and $V$ is number of vertices. Our implementation spends $0.0232$ seconds on the instance common-1-5000.txt with $n = 5000$

\subsection{References}
\begin{description}
  \item[1.] \emph{APLgraphlib---A library for Basic Graph Algorithms in APL}, version 2.11, 2016, Iverson Project, {\tt github.com/iverson/APLgraphlib}.\sidenote{If you use references to code, books, or papers, be professional about it. Use whatever style you want, but be consistent.}

  \item[2.] A. Lovelace, \emph{Algorithms and Data Structures in Pascal}, Addison--Wesley 1881. 
\end{description}

\end{document}