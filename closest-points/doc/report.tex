\documentclass{tufte-handout}
\usepackage{amsmath}
\usepackage[utf8]{inputenc}
\usepackage{mathpazo}
\usepackage{booktabs}
\usepackage{microtype}

\pagestyle{empty}


\title{Closest Pair Report}
\author{Michal Derdak, Tomas Vit, Kristinn þór Jónsson, Rafal Piotr Markiewicz, Alberto Martin Belloso, Rimniceanu Sabin and Albert Llebaria Holda}

\begin{document}
  \maketitle

  \section{Results}

  Our implementation produces the expected results on all input--output file pairs. 
  The current implementation has been developed with python3.

  The following table shows the closest pairs in the input files {\tt wc-instance-*.txt}.
  Here $n$ denotes the number of points in the input,
  and $(u,v)$ denotes a closest pair of points at distance $\delta$.

  \bigskip\noindent
  \begin{tabular}{rrrr}\toprule
    $n$ & $u$ & $v$ & $\delta$ \\\midrule
    2 & 0 & 1 & 1 \\
    6 & 0 & 0 & 1 \\
    14 & 0 & 0 & 1 \\
    30 & 0 & 0 & 1 \\
    62 & 0 & 0 & 1 \\
    126 & 0 & 0 & 1 \\
    254 & 0 & 0 & 1 \\
    1022 & 0 & 0 & 1 \\
    4094 & 0 & 0 & 1 \\
    16382 & 0 & 0 & 1 \\
    65534 & 0 & 0 & 1 \\
  \end{tabular}


  \section{Implementation details}
  We create two lists, one sorted by x and the other one by y ($P_x$ and $P_y$).
  For the next recursive calls, we find the middle position and value of x, and with this, we split        $P_y$ into two new lists ($Q_y$ and $R_y$). Once we have this, the next recursive calls will have as arguments the x-sorted lists splitted by the middle position ($Q_x$ for the left side and $R_x$ for the right side) and the two new y-sorted lists ($Q_y$ and $R_y$). 
  When the length of the curent x-sorted list is equal or smaller than 3, we find the minimum distance between its points, which will be the distances returned by the recursive calls, one for the left and one for the right. 
  After the recursion, we find the smallest distance of both left side points and right side points, called $\delta$.  

  A part from that distance we also need to find the minimum distance in the middle region ($S$ from the book). We construct the list $S$ sorted by y-coordinate: $S_y$. If there is only one point in this list, we return the previous smallest distance $\delta$. If not, we compute the minimum distance inspecting the 15 closest point to the current point or, if the length of $S_y$ is smaller than 15, we inspect all the points.
  
  Having the minimum distance in $S$ and $\delta$, we return the smallest one between these two, which will correspond to the closest points distance.

  Our running time is $O(n\log n)$ for $n$ points.


\end{document}
