\documentclass{tufte-handout}
\usepackage{amsmath}
\usepackage[utf8]{inputenc}
\usepackage{mathpazo}
\usepackage{booktabs}
\usepackage{microtype}

\pagestyle{empty}


\title{Flow Report}
\author{Bureau of Investigative Matters on behalf of the glorious Soviet Motherland, Group C}

\begin{document}
  \maketitle

  \section{Results}

  Our implementation successfully computes a flow of 163 on the input file, confirming the analysis of the Imperialist scum.

  We have analysed the possibilities of descreasing the capacities near Minsk.
  Our analysis is summarised in the following table:

\bigskip
  \begin{tabular}{rccc}\toprule
    Case & 4W--48 & 4W--49 & Effect on flow \\\midrule
    0& 30 & 30 & no change \\
    1& 10 & 10 & $-20$ \\
    2& 10 & 20 & $-10$ \\
    3& 20 & 10 & $-10$ \\
    4& 20 & 20 & no change \\
    5& 30 & 10 & no change \\
    6& 10 & 30 & no change \\
    7& 30 & 20 & no change \\
    8& 20 & 30 & no change \\
 \\\bottomrule
  \end{tabular}
\bigskip

 Bellow we analyze the findings:

\begin{tabular}{rccccc}\toprule
From&From id&To&To id&Flow&Capacity\\\midrule
2&1&45&19&6&6\\
2&1&46&18&12&12\\
3W&17&46&18&2&2\\
3E&16&4E&15&4&4\\
2S&2&6&13&24&24\\
52&10&8&9&16&16\\
52&10&7&11&4&4\\
52&10&51&25&2&2\\
R&27&b&26&8&8\\
10&7&7&11&10&10\\
7&11&51&25&28&28\\
7&11&50&24&34&34\\
5&12&49&23&12&12\\
46&18&47&22&20&20\\
4W&20&5&12&4&4\\
4W&20&49&23&10&10\\
4W&20&48&21&10&10\\
b&28&H&30&19&19\\
b&26&51&25&8&8\\
50&24&b&33&17&17\\
49&23&b&35&2&2\\
b&29&H&30&5&5\\
H&30&1&42&16&16\\
1&38&2&46&30&30\\
b&35&1&38&2&2\\
b&34&4&39&36&36\\
4&39&8&44&16&16\\
4&39&5&45&36&36\\
b&33&4&39&17&17\\
7&40&8&44&24&24\\
b&31&S&41&10&10\\
1&42&B&50&10&10\\
M&43&9&49&3&3\\
5&45&9&49&25&25\\
2&46&6&48&29&29\\
 \\\bottomrule
  \end{tabular}
\bigskip

  In case 1, the new bottleneck table becomes as above.

  \begin{tabular}{rccccc}\toprule
From&From id&To&To id&Flow&Capacity   \\\midrule
2   &1      &45&19   &6   &6          \\
2   &1      &46&18   &12  &12         \\
3W  &17     &46&18   &2   &2          \\
3W  &17     &4W&20   &34  &34         \\
3E  &16     &4E&15   &4   &4          \\
2S  &2      &6 &13   &24  &24         \\
52  &10     &8 &9    &16  &16         \\
52  &10     &7 &11   &4   &4          \\
52  &10     &51&25   &2   &2          \\
R   &27     &b &26   &8   &8          \\
10  &7      &7 &11   &10  &10         \\
7   &11     &51&25   &28  &28         \\
7   &11     &50&24   &34  &34         \\
5   &12     &49&23   &12  &12         \\
46  &18     &47&22   &20  &20         \\
4W  &20     &5 &12   &4   &4          \\
4W  &20     &49&23   &20  &20         \\
4W  &20     &48&21   &10  &10         \\
b   &28     &H &30   &19  &19         \\
b   &26     &51&25   &8   &8          \\
50  &24     &b &33   &17  &17         \\
49  &23     &b &35   &2   &2          \\
b   &29     &H &30   &5   &5          \\
H   &30     &1 &42   &16  &16         \\
1   &38     &2 &46   &30  &30         \\
b   &35     &1 &38   &2   &2          \\
b   &34     &4 &39   &36  &36         \\
4   &39     &8 &44   &16  &16         \\
4   &39     &5 &45   &36  &36         \\
b   &33     &4 &39   &17  &17         \\
b   &32     &7 &40   &29  &29         \\
7   &40     &8 &44   &24  &24         \\
b   &31     &S &41   &10  &10         \\
1   &42     &B &50   &10  &10         \\
M   &43     &9 &49   &3   &3          \\
5   &45     &9 &49   &25  &25         \\
2   &46     &6 &48   &29  &29         \\
 \\\bottomrule
  \end{tabular}
\bigskip

  In case 2, the new bottleneck becomes as above.

 \begin{tabular}{rccccc}\toprule
From&From id&To&To id&Flow&Capacity\\\midrule
2   &1      &45&19   &6   &6        \\
2   &1      &46&18   &12  &12       \\
3W  &17     &46&18   &2   &2        \\
3W  &17     &4W&20   &34  &34       \\
3E  &16     &4E&15   &4   &4        \\
2S  &2      &6 &13   &24  &24       \\
52  &10     &8 &9    &16  &16       \\
52  &10     &7 &11   &4   &4        \\
52  &10     &51&25   &2   &2        \\
R   &27     &b &26   &8   &8        \\
10  &7      &7 &11   &10  &10       \\
7   &11     &51&25   &28  &28       \\
7   &11     &50&24   &34  &34       \\
5   &12     &49&23   &12  &12       \\
46  &18     &47&22   &20  &20       \\
4W  &20     &5 &12   &4   &4        \\
4W  &20     &49&23   &10  &10       \\
4W  &20     &48&21   &20  &20       \\
b   &28     &H &30   &19  &19       \\
b   &26     &51&25   &8   &8        \\
50  &24     &b &33   &17  &17       \\
49  &23     &b &35   &2   &2        \\
b   &29     &H &30   &5   &5        \\
H   &30     &1 &42   &16  &16       \\
1   &38     &2 &46   &30  &30       \\
b   &35     &1 &38   &2   &2        \\
b   &34     &4 &39   &36  &36       \\
4   &39     &8 &44   &16  &16       \\
4   &39     &5 &45   &36  &36       \\
b   &33     &4 &39   &17  &17       \\
b   &32     &7 &40   &29  &29       \\
7   &40     &8 &44   &24  &24       \\
b   &31     &S &41   &10  &10       \\
1   &42     &B &50   &10  &10       \\
M   &43     &9 &49   &3   &3        \\
5   &45     &9 &49   &25  &25       \\
2   &46     &6 &48   &29  &29       \\
 \\\bottomrule
  \end{tabular}
  \bigskip

 In case 3, the new bottleneck becomes as above.

\bigskip
\bigskip
\bigskip
\bigskip

Our thorough investigation has uncovered that decreasing the Minsk capacity by up to 20 in total is acceptable and will not affect our Motherland's Glory. 

 Additionally, Independent sources suggest that the Party Secretary of Minsk has recently acquired a rare taste in gold. Capitalist gold. And from our findings it is clear that the Minsk infrastructure is struggling with a "financial deficit". Insinuating counter-revolutionary intent would be absurd of such an outstanding member of the Party. However, we recommend immediate and permanent disciplinary action be taken as a precaution.

  \section{Implementation details}

   We use an implemenation of Ford--Fulkerson's flow method in order to calculate the maximum flow, as described in Jon Kleinberg and Eva Tardos, \emph{Algorithm Design}, chap.~7.

  Since the railway is considered undirected, we have decided to represent it with an undirected graph that we treat as a directed graph in which each undirected edge is represented by two directed edges facing opposite directions and holding equal capacities.
We create the residual network by decreasing the flow of a reversed edge whenever a flow of an edge is increased.
As for the actual traversal of the graph, a simple Breadth First traversal was used.

\bigskip

  The running time is O(mnC). For graphs with m edges, n nodes, and integer capacities C.

\bigskip

  Our datatype for edge is this:

  \begin{verbatim}
class Edge:
    origin: int
    destination: int
    capacity: int
    flow: int
    is_reversed: bool
  \end{verbatim}


\end{document}
